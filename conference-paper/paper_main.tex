\RequirePackage{fix-cm}
\documentclass[conf]{new-aiaa}

\usepackage[utf8]{inputenc}
\usepackage{hyperref}
\usepackage{graphicx}
\usepackage{siunitx}
\sisetup{group-separator = {,}}
\usepackage{booktabs}
\usepackage{enumitem}
\usepackage{float}
\usepackage{amsmath}
\usepackage[version=4]{mhchem}
\usepackage{longtable,tabularx}
\usepackage{placeins}
\usepackage{multirow}
\usepackage{booktabs}
\usepackage{latexsym}
\usepackage{subcaption}
\usepackage{latexsym}
\usepackage[sort&compress,numbers]{natbib} % For bibtex \citet, \citep
\usepackage{hypernat} % To get natbib to play nicely with hyperref
\usepackage{doi} % For getting hyperlinked DOI in the references


\hypersetup{
    pdfauthor={Peter Atma}, % insert author here
	pdftitle={Aviation 2022 Abstract}, % insert title here
	pdfsubject={Propulsion Analysis and Optimization of an Ultra Bypass Turbofan Engine}, % insert keywords here
}

\graphicspath{{../figures/}}

\title{Comparing Hydrogen and Jet-A for a Ultra High-Bypass Turbofan with Water Recirculation} % change

\author{Peter N. Atma\footnote{MSE Student, Department of Aerospace Engineering, AIAA Student Member}}
\author{Andrew H.R. Lamkin\footnote{Ph.D.~Candidate, Department of Aerospace Engineering, AIAA Student Member}}
\author{Joaquim R.~R.~A.~Martins\footnote{Professor, Department of Aerospace Engineering, AIAA Fellow}}
\affil{University of Michigan, Ann Arbor, MI, 48109}

% ==================================================
%	Abstract
% ==================================================

% ==================================================
\begin{document}

\maketitle

\begin{abstract}
    Advances in commercial propulsion technology have resulted in the creation of larger and more advanced high-bypass turbofan engines with higher bypass ratios, pressure ratios, and internal temperatures.
    With the focus on reducing aircraft emissions, aircraft and engine manufacturers have been pushing for hydrogen as a fuel and more efficient engines to lower carbon emissions.
    Carrying and burning hydrogen introduces complexity and weight penalties that can be partially offset by taking advantage of the fuel's thermodynamic and chemical properties.
    In this study, we compare the benefits of using hydrogen against Jet-A fuel in a high-bypass turbofan with a closed-loop water recirculation system.
    We will setup and perform a gradient-based optimization parameter sweep to understand the performance and emissions benefits of water recirculation using Jet-A and hydrogen fuels.
    This study will demonstrate the benefits of hydrogen combustion with water recirculation in order to encourage next-generation propulsion systems to exploit the properties of hydrogen for improved performance and reduced emissions.
\end{abstract}

\section{Introduction}
% Message 1: Motivation to incorporate low-emission fuels and techniques
The effects of climate change are pushing the aviation industry towards hydrogen-fueled propulsion systems as a solution to reduce emissions.
N+3 technology estimates for turbofan engines that burn hydrocarbon fuels suggest that higher efficiency can be achieved by designing ultra high-bypass (UHB) engines with small cores and high overall pressure ratios (OPR).
Higher OPR and smaller cores are pushing the limits of compressor and turbine design, placing an upper bound on potential performance and emissions improvements.
Switching to hydrogen as the primary fuel source reduces carbon dioxide emissions immediately, but adds complexity and weight that offset the benefits.
However, hydrogen is a versatile fuel with advantageous chemical and thermodynamic properties that can be leveraged to further increase the performance and emission reduction of the propulsion system.
In this study, we investigate the tradeoff between burning hydrocarbon fuels like Jet-A versus hydrogen for UHB ratio turbofan engines.
We will look at new ideas with water recirculation to further squeeze performance and efficiency from Jet-A and hydrogen combustion using gradient-based design optimization.

% Message 2: Background and references to support using H2 and water injection in HBTF engines
Water recirculation is the process of extracting water from the exhaust stream of a propulsion system and injecting it upstream of the combustor as finely atomized droplets.
NASA, Boeing, and Rolls-Royce studied this concept and suggested that this technique reduces the NOx emissions as much as 47 percent~\cite{nasa_inject}.
Additionally, water recirculation improves fuel efficiency and thrust output with lower burner temperatures that can improve the lifetime of turbine blades and reduce noise~\cite{nasa_inject}.
Traditional propulsion systems that burn hydrocarbon fuels would require external water storage on the aircraft because they do not produce enough in the exhaust for recirculation.
The added weight of tanks, pumping, and ducting makes this concept infeasible for a conventional aircraft.
However, if hydrogen fuels are used, the main product of hydrogen combustion is water vapor and can thus be recovered.
The ability to recirculate water vapor from the exhaust of hydrogen combustion reduces the requirement for storage tanks and allows for the creation of a closed loop system inside the propulsion cycle.

Zero-dimensional cycle modeling is an efficient tool for predicting the initial design, performance, and emissions of new propulsion concepts.
Zero-dimensional cycle analysis uses a first-principles approach with a chemical equilibrium analysis (CEA) thermodynamics solver that considers the molecular species of different fuels.
This can be used to understand the trends and design limitations of an engine cycle and study potential improvements in design using optimization.
The industry standard for zero and one-dimensional cycle analysis is the Numerical Proprulsion System Simulation (NPSS) framework~\cite{JonesNPSS}.
NPSS is a modular object-oriented framework that models engine components as individual blocks with several thermodynamic solvers.
\citet{Hendricks2019} and \citet{Gray2017b} created a tool called pyCycle with the same functionality as NPSS, but includes analytical derivatives for each engine component and thermodynamic solver.
We use pyCycle because it is built on top of the OpenMDAO framework~\cite{Gray2019a} to simplify gradient-based optimization and leverage hierarchical nonlinear solver structures for robustness.

% Message 3: Introduce the extension of the HBTF and propose novel contributions
In this work, we will analyze the potential propulsion benefits of a closed-loop water vapor recovery and water injection system in a high-bypass turbofan engine by optimizing the thrust-specific energy consumption (TSEC).
TSEC is the thrust specific fuel consumption multiplied by the lower heating value of the fuel and is a metric for comparing the efficiency of different fuel choices.
We will develop new pyCycle components for water injection and vapor recovery within pyCycle to understand the benefit of a closed loop recirculation system.
Finally, we will present a comparison between Jet-A and hydrogen combustion with and without water recirculation in terms of improvement of TSEC.

This work is organized as follows. First, in Section \ref{sec:method}, we introduce the turbofan model and explain how we will implement the water injection and water recovery components in pyCycle.
In section \ref{sec:results} explains the implementation of the multi-point optimization problem and the conditions limiting the engine model.

% \section{Model Description}
% \label{sec:epModel}
% % Engine Architecture: Describe the flow path of the engine and establish the mechanical coupling.
% We will use the NASA advanced technology UHB geared turbofan engine cycle, referred to as the "N+3" engine \cite{Jones2017a} for this work.
% The N+3 reference cycle represents a UHB ratio geared turbofan that could be available in the 2030–2040 time frame and was selected since it has already been modeled in pyCycle as a robust example cycle and includes the advanced engine cycle improvements such as a large bypass ratio, high pressure ratio, and high combustion temperature.
% The flow path consists of an inlet that directs ambient air through a fan, followed by a duct that splits the flow into a core flow and a bypass flow, each of which ends in a bypass nozzle and core nozzle, respectively.
% The fan and low pressure compressor (LPC) are connected to the low pressure turbine (LPT) by the low pressure shaft and the high pressure compressor (HPC) is connected to the high pressure turbine (HPT) by the high pressure shaft.
% Along the axial flow path, the zero-dimensional thermodynamic connections are solved using CEA to ensure first the principle governing equations are satisfied.
% The closed-loop water recovery system is implemented into the cycle by placing a water injection component directly before the HPC.
% This component will inject pure water into the core stream which will reduce the combustion temperature due to heat absorption from vaporization.
% The water vapor recovery component is placed directly before the core nozzle to extract water from the core stream and recycle it back to the water injector.
% In practice a water vapor recovery component requires a complex condenser model in the core stream, this work will only look at the potential benefits of extracting water vapor and not the recovery method specifically.
% The component flow interface and mechanical connections, including the water injector and water extractor, are depicted in Figure \ref{fig:hbtf_cycle}.

% \begin{figure}[!hbt]
%     \centering
%     \includegraphics[width=0.75\textwidth]{turbofan_wvr.pdf}
%     \caption{The configuration of a high-bypass turbofan model with an integrated closed-loop water vapor recovery and injection system.
%         Water vapor is recovered from the core stream of the engine before the nozzle and is re-injected into the core stream before the HPC.}
%     \label{fig:hbtf_cycle}
% \end{figure}

% % Preliminary Results
% We tested the N+3 propulsion model with water injection using Jet-A and liquid hydrogen fuels to determine if there are any immediate improvements in efficiency.
% The preliminary results from this analysis are shown in Figure \ref{fig:results} and show the relative improvement in TSEC with water injection for the same engine design with each type of fuel.
% We see improvements in the cruise engine efficiency using a baseline analysis and will be further improved using design optimization.

% \begin{figure}[!hbt]
%     \centering
%     \includegraphics[width=0.8\textwidth]{JetA-H2_bar_chart_diff.pdf}
%     \caption{Percent relative improvement in thrust specific energy consumption (TSEC) of an ultra-high bypass turbofan engine from water injection.
%         Data was collected with and without water injected into the core stream of the engine for Jet-A and hydrogen fuels.}
%     \label{fig:results}
% \end{figure}

\section{Methodology}
\label{sec:method}
\subsection{Engine Model Overview}
The propulsion model is constructed in pyCycle which is built on top of OpenMDAO to allow a modular design of the engine components and the coupling of other engineering disciplines for analysis \cite{Gray2019a}.
The engine model that was selected for this project is the NASA advanced technology UHB geared turbofan engine cycle, referred to as the "N+3" engine \cite{Jones2017a}.
The N+3 reference cycle represents a UHB ratio geared turbofan that could be available in the 2030–2040 time frame and was selected since it has already been modeled in pyCycle as a robust example cycle and includes the advanced engine cycle improvements such as a large bypass ratio, high pressure ratio, and high combustion temperature.
The flow path consists of an inlet that directs ambient air through a fan, followed by a duct that splits the flow into a core flow and a bypass flow, each of which ends in a bypass nozzle and core nozzle, respectively.
The fan and low pressure compressor (LPC) are connected to the low pressure turbine (LPT) by the low pressure shaft and the high pressure compressor (HPC) is connected to the high pressure turbine (HPT) by the high pressure shaft.
Along the axial flow path, the zero-dimensional thermodynamic connections are solved using CEA to ensure the first principle governing equations are satisfied.
Figure \ref{fig:N3_original} shows how the 25 elements in the N+3 cycle are connected together with the fluid flows in blue, the mechanical connections in red, and the performance elements in green \cite{Hendricks2019}.

\begin{figure}[!hbt]
    \centering
    \includegraphics[width=0.75\textwidth]{N3_diagram.pdf}
    \caption{
        N+3 engine cycle layout that shows how each of the pyCycle components are connected and how performance metrics are computed.
        The fluid flow connections are shown in blue, the mechanical connections are shown in red, and the performance connections are shown in green.
    }
    \label{fig:N3_original}
\end{figure}

\noindent
The N+3 engine model as it is implemented in pyCycle has many complex thermodynamic and performance connections between different operating conditions.
These operating conditions are top-of-climb (TOC), rotating takeoff (RTO), sea-level static (SLS), and cruise (CRZ).
To account for each of these operating conditions, the N+3 model uses a technique called Multipoint Design Point (MDP) modeling to converge the model to an engine design that satisfies the requirements at each of these operation conditions.
One of the operation conditions is specified as the design point, which in this case is TOC, and the other operating conditions are the off-design points.
The design point of the engine model is converged using a linear Newton solver which then passes the geometric sizing variables such as cross-sectional areas and map scalars to the off-design points and similarly converges those.
Then, a nonlinear Newton solver is used to converge the overall model with respect to the variable connections between each operating condition.
An XDSM diagram of the nominal N+3 engine model is shown in Figure \ref{fig:N3_xdsm} \cite{Hendricks2019}.

\begin{figure}[!hbt]
    \centering
    \includegraphics[width=0.75\textwidth]{N3_xdsm.pdf}
    \caption{
        N+3 engine multipoint setup XDSM diagram.
        This XDSM diagram shows how each of the operating conditions is coupled with each other and how the outputs are computed.
    }
    \label{fig:N3_xdsm}
\end{figure}

\subsection{Water Recovery Model}
The closed-loop water recovery system is implemented into the cycle by placing a water injection component directly before the HPC.
This component will inject pure water into the core stream which will reduce the combustion temperature due to heat absorption from vaporization.
This location was chosen as it was one of the configurations chosen by the afformetioned NASA, Boeing, and Rolls-Royce study that did not require direct combustor injection \cite{nasa_inject}.
The water vapor recovery component is placed directly before the core nozzle to extract water from the core stream and recycle it back to the water injector.
Water is recovered from this location as it is the last possible location before the nozzle in which water can be extracted.
In practice a water vapor recovery component requires a complex condenser model in the core stream, this work will only look at the potential benefits of extracting water vapor and not the recovery method specifically.
The component flow interface and mechanical connections, including the water injector and water extractor, are depicted in Figure \ref{fig:hbtf_cycle}.

\begin{figure}[!hbt]
    \centering
    \includegraphics[width=0.75\textwidth]{turbofan_wvr.pdf}
    \caption{
        The configuration of a high-bypass turbofan model with an integrated closed-loop water vapor recovery and injection system.
        The water vapor recovery system (extractor) extracts a fraction of the water in the core stream and reinjects it upstream of the high-pressure compressor.
        This diagram illustrates the feedback effect that this implementation has on the overall core flow.}
    \label{fig:hbtf_cycle}
\end{figure}

\noindent
To model this complicated water recirculation loop in pyCycle new components had to be developed.
The two new components that were developed are a water injector to add water to the flow and a water extractor to divert a fraction of the water in the flow back upstream.
The injector component is based on the combustor component already included in pyCycle which operates by injecting a given fuel-to-air ratio (FAR) or fuel flow rate to determine the mass of fuel added to the incoming flow.
This new mixture combination is then used to compute the various chemical species present in the flow at the given thermodynamic state which is determined by the incoming flow.
The new species composition and thermodynamic variables are then determined using a Gibbs Free Energy minization approach called Chemical Equilibrium with Applications (CEA).
Once the new composition is determined, the various thermodynamic variables are passed to the next component in the engine.
The water injector component was created to work similarly to the combustor but would instead inject water as a reactant instead of the fuel.
The water injector can take either water-to-air ratio (WAR) or water mass flow rate to determine how much water is added to the flow.
A simple schematic of the injector is shown in Figure \ref{fig:injector} where $Y_{H_2O}$ is the mole fraction of water molecules.

\begin{figure}[!hbt]
    \centering
    \includegraphics[width=0.75\textwidth]{injector.pdf}
    \caption{
        Injector component schematic.
        Water from the extractor is simply injected into the core flow upstream of the high-pressure compressor.
    }
    \label{fig:injector}
\end{figure}

\noindent
The water extractor component is based on an air-bleed component in pyCycle with the added complexity of extracting a fraction of a specific species within the flow.
This extractor component works by first determining the water content of the incoming flow and extracting a given fraction of that water flow.
The water content of the incoming flow is determined by solving a CEA analysis of the incoming flow.
This water content in the incoming flow is then used with a given fraction of the water to extract to determine the how much water mass flow to remove from the stream.
At the same time, the atomic mixture of the core stream is updated to represent the mixture that would be left if a number of hydrogen and oxygen atoms corresponding to the amount of water extracted were removed.
The outputs of the extractor component are the flow rate of the core flow and the corresponding flow properties in addition to the extracted water mass flow rate.
While this process certainly would have pressure loss penalties in terms of condensing water in a condersor, this project just looks at the effects of water recovery since there are no current models for nozzle exhaust condensors available.
A simple schematic of the extractor is shown in Figure \ref{fig:extractor} where $Y_{H_2O}$ is the mole fraction of water molecules and $X_{H_2O,k}$ is the fraction of water that is recovered from the core stream of operating condition, $k$.

\begin{figure}[!hbt]
    \centering
    \includegraphics[width=0.75\textwidth]{extractor.pdf}
    \caption{
        Extractor component schematic.
        A certain mole fraction of the flow coming out of duct5 is comprised of water of which a certain specified fraction is extracted.
        The extracted water is routed back upstream to the injector and the rest is simply exhausted out the nozzle.
    }
    \label{fig:extractor}
\end{figure}

\noindent
Unlike other flow streams in the N+3 model, the vapor recovery loop has a feedback effect since it is adding a mass flow rate at the injector and extracting a mass flow rate at the extractor.
This is because the pyCycle model is solved sequentially during each Newton iteration.
So a Nonlinear Block Gauss-Seidel solver would be sufficient to solve this feedback loop.
However, the N+3 model already has a nonlinear Newton solver at the Multipoint Design level, so the extractor mass flow rate and injector mass flow rate variables are connected for each design point and use this nonlinear solver to converge the water streams.
These connections can be seen in Figure \ref{fig:N3_xdsm_full}

% Talk about how areas are connected in pyCycle (injector/extractor set at CRZ everything else at TOC)

\subsection{Performance Metrics}
For measuring the efficiency of jet engines, thrust-specific fuel consumption (TSFC) is generally used since is a metric of how low the fuel burn is for a given level of thrust.
However, when comparing Jet-A and hydrogen fuels this is not such a good metric since a given mass flow rate of hydrogen has an energy content almost 3 times that of Jet-A.
Therefore, a new metric for comparing the relative efficiencies of engine running on Jet-A versus hydrogen is thrust-specific energy consumption (TSEC) which multiplies TSFC by the lower heating value (LHV) of the fuel \cite{Adler2022d}.

\begin{equation}
    TSFC = \frac{\Dot{m}_{fuel}}{F_{thrust}}
\end{equation}

\begin{equation}
    TSEC = \frac{\Dot{m}_{fuel} LHV}{F_{thrust}} = TSFC \times LHV
\end{equation}

\noindent
With all of the sub-models of the N3 engine presented, the complete engine cycle XDSM diagram is shown in Figure \ref{fig:N3_xdsm_full}.

\begin{figure}[!hbt]
    \centering
    \includegraphics[width=0.75\textwidth]{N3_xdsm_full.pdf}
    \caption{
        Full model N+3 XDSM diagram.
        This XDSM diagram shows the multipoint coupling between the different operation conditions and shows how the water recovery fractions are used to solve for water mass flow rates.
    }
    \label{fig:N3_xdsm_full}
\end{figure}

\subsection{Model Parametric Studies}
To understand the coupling between the various operation conditions and design point, a few simple parametric studies of the water recovery fractions were run.
The design parameters for the engine are shown in the table below.
Additionally, the humidity conditions listed below are used in the flight conditions component in the N+3 engine to ensure that the humidity of the air entering the engine is equal to the humidity of the atmosphere at a given operating condition.

\begin{table}[H]
    \centering
    \caption{
        N3 engine model parameters.
        These parameters specify the engine design used for analysis in this report.
        Atmospheric parameters are presented that specify the flight conditions and humidity of the atmosphere.}
    \begin{tabular}{|c|c|c|c|}
        \hline
        Parameter             & Value    & Units                 & Comments                                       \\
        \hline
        $W_{TOC}$             & 820.441  & $lbm/s$               & TOC total mass flow rate of air                \\
        $BPR_{TOC}$           & 23.945   & $-$                   & TOC bypass Ratio                               \\
        $PR_{HPC,TOC}$        & 53.633   & $-$                   & TOC high-pressure compressor pressure ratio    \\
        $PR_{LPC,TOC}$        & 3.00     & $-$                   & TOC low-pressure compressor pressure ratio     \\
        $PR_{fan,TOC}$        & 1.30     & $-$                   & TOC fan pressure ratio                         \\
        $F_{net,SLS}$         & 28620.84 & $lbf$                 & SLS net thrust                                 \\
        $F_{net,CRZ}$         & 5510.72  & $lbf$                 & CRZ net thrust                                 \\
        $T_{4,TOC}/T_{4,RTO}$ & 0.926    & $-$                   & TOC-to-RTO temperature ratio                   \\
        $T_{4,RTO}$           & 3400.0   & $^{\circ}R$           & RTO combustor temperature                      \\
        $h_{TOC}$             & 0.001    & $kg_{water}/kg_{air}$ & humidity raio at TOC                           \\
        $h_{RTO}$             & 0.007    & $kg_{water}/kg_{air}$ & humidity raio at RTO                           \\
        $h_{SLS}$             & 0.007    & $kg_{water}/kg_{air}$ & humidity raio at SLS                           \\
        $h_{CRZ}$             & 0.001    & $kg_{water}/kg_{air}$ & humidity raio at CRZ                           \\
        $LHV_{JetA}$          & 18564.0  & $BTU/lbm$             & Lower heating value of Jet-A \cite{boeingJetA} \\
        $LHV_{H2}$            & 51591.0  & $BTU/lbm$             & Lower heating value of H2 \cite{engtoolboxH2}  \\
        \hline
    \end{tabular}
    \label{engine_params}
\end{table}

\subsection{Optimization Problem}
The optimization problem of the N3 engine and the vapor recovery loop are shown in Figure \ref{fig:N3_xdsm_opt}.

\begin{figure}[!hbt]
    \centering
    \includegraphics[width=0.75\textwidth]{N3_xdsm_opt.pdf}
    \caption{
        XDSM diagram of the multipoint optimization problem with constraints.
        The XDSM diagram shows the variables and outputs within the optimization model and how each of these values is connected to the optimizer and design points.}
    \label{fig:N3_xdsm_opt}
\end{figure}

\noindent
This optimization problem was run for both Jet-A and H2 fuels using the parameters given in Table \ref{engine_params}.
The optimization problem for each fuel was specified as:

\begin{table}[h]
    \centering
    \caption{
        Multipoint optimization problem definition.
        The 4 design variables are the water recovery fractions of the core exhaust stream at each operation condition.
        The objective function is the thrust specific energy consumption at the cruise condition with a net thrust constraint at the design point, TOC.
    }
    \small
    \renewcommand{\arraystretch}{1.2}
    \begin{tabular}{r l l l l}
        \toprule
                        & Variable/Function           & Description                                                & Units                 & Quantity \\
        \hline
        minimize        & $ \text{TSEC}_{CRZ} $       & Thrust specific energy consumption at the cruise condition & $\frac{BTU}{hr-lb_f}$ & 1        \\
                        &                             &                                                            &                       &          \\
        with respect to & $x_{H2O,TOC}$               & Water recovery fraction at TOC                             & -                     & 1        \\
                        & $x_{H2O,RTO}$               & Water recovery fraction at RTO                             & -                     & 1        \\
                        & $x_{H2O,SLS}$               & Water recovery fraction at SLS                             & -                     & 1        \\
                        & $x_{H2O,CRZ}$               & Water recovery fraction at CRZ                             & -                     & 1        \\
                        & $PR_{HPC,TOC}$              & TOC high-pressure compressor pressure ratio                & -                     & 1        \\
                        & $PR_{LPC,TOC}$              & TOC low-pressure compressor pressure ratio                 & -                     & 1        \\
                        & $PR_{fan,TOC}$              & TOC fan pressure ratio                                     & -                     & 1        \\
                        & $T_{4,TOC}/T_{4,RTO}$       & TOC-to-RTO temperature ratio                               & -                     & 1        \\
                        & $T_{4,RTO}$                 & RTO combustor temperature                                  & $^\circ$R             & 1        \\
        \cline{3-5}
                        &                             & Total                                                      &                       & 4        \\
                        &                             &                                                            &                       &          \\
        subject to      & $F_{net,TOC} \geq 5800$ lbf & Target net thrust at TOC                                   & lbf                   & 1        \\
                        & $D_{Fan} \leq 100$ $in^2$   & Maximum Fan Diameter                                       & $in^2$                & 1        \\
        \cline{3-5}
                        &                             & Total                                                      &                       & 1        \\
        \bottomrule
    \end{tabular}
    \label{tab:opt_problem}
\end{table}


% Briefly discuss the optimization problem
We will perform gradient-based design optimization of the engine model with a multi-point architecture that considers several flight conditions experienced by commercial aircraft.
An XDSM diagram of the optimization problem with the multipoint formulation is shown in Figure \ref{fig:opt_prob}.

\begin{figure}[!hbt]
    \centering
    \includegraphics[width=0.75\textwidth]{N3_inject.pdf}
    \caption{XDSM diagram of the multi-point optimization problem with constraints.
        The XDSM diagram shows the variables and outputs within the optimization model and how each of these values is connected to the optimizer and design points.}
    \label{fig:opt_prob}
\end{figure}

The objective is to minimize the TSEC at a cruise condition subject to constraints, thrust, max temperature, and engine diameter.
We will perform a parameter sweep of the limiting constraint to understand the impact of different emissions and performance requirements on the design space.

\section{Results}
\label{sec:results}

% results of parameter sweep
% results of parameter sweep
% results of optimization problems

\section{Conclusion}
\label{sec:conc}
% Message 1: Re-state the problem
In this work, we will model, analyze, and optimize a UHB next-generation turbofan model considering Jet-A and hydrogen fuel using a novel closed-loop water vapor recovery to improve cycle efficiency.
% Message 2: Summarize the HBTF model
The high-bypass turbofan engine will be composed of separate sub-components modeled in pyCycle with the new water recovery and injection capability to assess the performance and emissions improvements.
We will compare Jet-A and liquid hydrogen with and without the closed-loop water vapor recovery system to determine the benefits of this technology implementation.
The results will show a quantitative comparison at the propulsion system level, with a qualitative assessment of the feasibility of water recirculation for both Jet-A and hydrogen fuels.
% Message 3: Expected results
Finally, this work will demonstrate the ability of gradient-based optimization with zero-dimensional cycle analysis to explore the potential benefits of hydrogen combustion with a closed-loop water vapor recovery and injection.

\bibliography{mdolab,references}

\end{document}